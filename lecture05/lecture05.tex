% Cornell CS 2043: Unix Tools and Scripting
% Spring 2016
%
% http://cs2043-sp16.github.io/
%
% These slides have been created using the metroplis theme demo found here:
%
%     https://github.com/matze/mtheme.git
%
% I am a huge fan of the metropolis theme...thanks @matze!

\documentclass[11pt]{beamer}
\usetheme{metropolis}
\metroset{background=dark} % easier on the eyes, in my opinion
\setbeamertemplate{bibliography item}{\insertbiblabel}

% linking
\usepackage{hyperref}

% Nicer tables
\usepackage{booktabs}
\usepackage[scale=2]{ccicons}

% finer control over enumerations
% example:
%   \begin{enumerate}[(1)]
%     \item first
%     \item second
%     \item ...
%   \end{enumerate}
%
%   and also
%
%   \begin{enumerate}[a)]
%   \begin{enumerate}[i.]
%   \begin{enumerate}[$\mathbb{R}$\textsuperscript{\arabic{enumi}}]
%
% and so on
\usepackage{enumerate}

%
% sultry code listings
%
\usepackage{minted}
\BeforeBeginEnvironment{minted}{\begin{block}{}} % These surround minted code with beamer blocks
\AfterEndEnvironment{minted}{\end{block}}        % to accentuate them

%
% Colors thanks to various themes from Monokai: http://www.colourlovers.com/lover/Monokai
%
\definecolor{make-it-concrete}{rgb}{0.670588, 0.666667, 0.596078}      % (171, 170, 152) / 255
\definecolor{cheap-carpet}{rgb}{0.145098, 0.196078, 0.200000}% (37,50,51) / 255
\definecolor{hairball}{rgb}{0.403922, 0.411765, 0.380392}% (103,105,97) / 255

%
% General slide coloring
%
\setbeamercolor{frametitle}{% adjusts too-white colors from beamer-metropolis
  bg=hairball!66,
  fg=cheap-carpet
}
\setbeamercolor{normal text}{
  bg=mDarkTeal,   % note: defined by beamer-metropolis
  fg=hairball!44
}
%
% beamer-block shaping and colors
%
\setbeamertemplate{blocks}[rounded]
\setbeamercolor{block title}{
  bg=make-it-concrete!66,
  fg=cheap-carpet
}
\setbeamercolor{block body}{
  bg=cheap-carpet,
  fg=make-it-concrete!66
}

%
% Custom commands
%
\newcommand{\colbf}[1]{\textcolor{mLightBrown!77!black}{#1}}% bold with color; mLightBrown defined by beamer-metropolis
\newcommand{\tsub}{\textsubscript}                          % more compact
\newcommand{\tsup}{\textsuperscript}                        % " " " " " "
\def\wl{\par \vspace{\baselineskip}}                        % force newline; MANY thanks:
                                                            %   https://groups.google.com/d/msg/comp.text.tex/abxw5nTALzg/puXHnyx6pM8J

%%%%%%%%%%%%%%%%%%%%%%%%%%%%%%%%%%%%%%%%%%%%%%%%%%%%%%%%%%%%%%%%%%%%%%%%%%%%%%%%%%%%%%%%%%%%%%%%%%%%%%%%%%%%%%%
%%%%%%%%%%%%%%%%%%%%%%%%%%%%%%%%%%%%%%%%%%%%%%%%%%%%%%%%%%%%%%%%%%%%%%%%%%%%%%%%%%%%%%%%%%%%%%%%%%%%%%%%%%%%%%%
%%%%%%%%%%%%%%%%%%%%%%%%%%%%%%%%%%%%%%%%%%%%%%%%%%%%%%%%%%%%%%%%%%%%%%%%%%%%%%%%%%%%%%%%%%%%%%%%%%%%%%%%%%%%%%%
%%%%%%%%%%%%%%%%%%%%%%%%%%%%%%%%%%%%%%%%%%%%%%%%%%%%%%%%%%%%%%%%%%%%%%%%%%%%%%%%%%%%%%%%%%%%%%%%%%%%%%%%%%%%%%%
%
% \begin
%
%%%%%%%%%%%%%%%%%%%%%%%%%%%%%%%%%%%%%%%%%%%%%%%%%%%%%%%%%%%%%%%%%%%%%%%%%%%%%%%%%%%%%%%%%%%%%%%%%%%%%%%%%%%%%%%
%%%%%%%%%%%%%%%%%%%%%%%%%%%%%%%%%%%%%%%%%%%%%%%%%%%%%%%%%%%%%%%%%%%%%%%%%%%%%%%%%%%%%%%%%%%%%%%%%%%%%%%%%%%%%%%
%%%%%%%%%%%%%%%%%%%%%%%%%%%%%%%%%%%%%%%%%%%%%%%%%%%%%%%%%%%%%%%%%%%%%%%%%%%%%%%%%%%%%%%%%%%%%%%%%%%%%%%%%%%%%%%
%%%%%%%%%%%%%%%%%%%%%%%%%%%%%%%%%%%%%%%%%%%%%%%%%%%%%%%%%%%%%%%%%%%%%%%%%%%%%%%%%%%%%%%%%%%%%%%%%%%%%%%%%%%%%%%
\title{05 \-- Expansions and Regular Expressions}
\subtitle{CS 2043: Unix Tools and Scripting, Spring 2016 \cite{prevSemesters}}
\date{February 5th, 2016}
\author{Stephen McDowell}
\institute{Cornell University}

\begin{document}
\maketitle

% TOC
\begin{frame}{Table of contents}
  \setbeamertemplate{section in toc}[sections numbered]
  \tableofcontents[hideallsubsections]
\end{frame}

\begin{frame}{Some Logistics}
  \begin{itemize}[<+- | alert@+>]
    \item The \texttt{assignments} repository on GitHub
    \item Course pacing...
    \item HW1 tonight
  \end{itemize}
\end{frame}

%
%%
%%%
%%%%
%%%%% section shell_expansion
%%%%%
%%%%%
%%%%%
%%%%%
\section{Shell Expansion}
\label{sec:shell_expansion}

\begin{frame}[fragile]{Wildcards}
  There are various special characters you have access too in your shell to expand phrases to match patterns,
  such as \texttt{*}, \texttt{?}, \texttt{\^{}}, \texttt{\{}, \texttt{\}}, \texttt{[}, \texttt{]}.
  \begin{itemize}[<+- | alert@+>]
    \item Any string
    \item A single character
    \item A phrase
    \item A restricted set of characters
  \end{itemize}
\end{frame}

\begin{frame}[fragile]{Shell Expansion: Example}
  \begin{itemize}
    \item \texttt{*} matches any string, including the null string (e.g. 0 or more characters)
  \end{itemize}
  \begin{center}
    {\small
    \begin{tabular}{|l|l|l|}
      \hline
      Input & Matched & Not Matched\\ \hline
      \texttt{Lec*} & \texttt{Lecture1.pdf Lec.avi} & \texttt{AlecBaldwin/}\\ \hline
      \texttt{L*ure*} & \texttt{Lecture2.pdf Lectures/} & \texttt{sure.txt}\\ \hline
      \texttt{*.tex} & \texttt{Lecture1.tex Presentation.tex} & \texttt{tex/}\\ \hline
    \end{tabular}
    }
  \end{center}
\end{frame}

\begin{frame}[fragile]{Shell Expansion: Example}
  \begin{itemize}
    \item \texttt{?} matches a single character
  \end{itemize}
  \begin{center}
    {\small
    \begin{tabular}{|l|l|l|}
      \hline
      Input & Matched & Not Matched\\ \hline
      \texttt{Lec?.pdf} & \texttt{Lec1.pdf Lec2.pdf} & \texttt{Lec11.pdf}\\ \hline
      \texttt{ca?} & \texttt{cat can cap} & \texttt{ca cake}\\ \hline
    \end{tabular}
    }
  \end{center}
\end{frame}

\begin{frame}[fragile]{Shell Expansion: Example}
  \begin{itemize}
    \item \texttt{[...]} matches any character inside the square brackets
    \begin{itemize}
      \item Use a dash to indicate a range of characters
      \item Can put commas between characters / ranges
    \end{itemize}
  \end{itemize}
  \begin{center}
    {\small
    \begin{tabular}{|l|l|l|}
      \hline
      Input & Matched & Not Matched\\ \hline
      \texttt{[SL]ec*} & \texttt{Lecture Section} & \texttt{Vector.tex}\\ \hline
      \texttt{Day[1-3]} & \texttt{Day1 Day2 Day3} & \texttt{Day5}\\ \hline
      \texttt{[A-Z,a-z][0-9].mp3} & \texttt{A9.mp3 z4.mp3} & \texttt{Bz2.mp3 9a.mp3}\\ \hline
    \end{tabular}
    }
  \end{center}
\end{frame}

\begin{frame}[fragile]{Shell Expansion: Example}
  \begin{itemize}
    \item \texttt{[\^{}...]} matches any character \colbf{not} inside the square brackets
  \end{itemize}
  \begin{center}
    {\small
    \begin{tabular}{|l|l|l|}
      \hline
      Input & Matched & Not Matched\\ \hline
      \texttt{[\^{}A-P]ec*} & \texttt{Section.pdf} & \texttt{Lecture.pdf}\\ \hline
      \texttt{[\^{}A-Za-z]*} & \texttt{9Days.avi} & \texttt{vacation.jpg}\\ \hline
    \end{tabular}
    }
  \end{center}
\end{frame}

\begin{frame}[fragile]{Shell Expansion: Example}
  \begin{itemize}
    \item \colbf{Brace Expansion}: \texttt{\{..., ...\}} matches any phrase inside the comma-separated braces.
    \item Suports ranges as well!
    \item Brace expansion needs at least two options to choose from.
  \end{itemize}\vspace*{-2ex}
  \begin{center}
    {\small
    \begin{tabular}{|l|l|}
      \hline
      Input & Matched\\ \hline
      \texttt{\{Hello,Goodbye\}\textbackslash World} & \texttt{Hello World Goodbye World}\\ \hline
      \texttt{\{Hi,Bye,Cruel\}\textbackslash World} & \texttt{Hi World By World Cruel World}\\ \hline
      \texttt{\{a..t\}} & Expands to the range \texttt{a} ... \texttt{t}\\ \hline
      \texttt{\{1..99\}} & Expands to the range \texttt{1} ... \texttt{99}\\ \hline
    \end{tabular}
    }
  \end{center}
  \colbf{Note}: NO SPACES.
  We haven't covered loops yet...but this is most useful when you want to do something like
  \begin{itemize}
    \item \texttt{for x in {1..99}; do echo \$x; done}
  \end{itemize}
\end{frame}

\begin{frame}[fragile]{Combining Them}
  Of course, you can combine all of these!

  \begin{center}
    {\small
    \begin{tabular}{|l|l|l|}
      \hline
      Input & Matched & Not Matched\\ \hline
      \texttt{*h[0-9]*} & \texttt{h3 h3llo.txt} & \texttt{hello.txt}\\ \hline
      \texttt{[bf][ao][row].mp?} & \texttt{bar.mp3 foo.mpg} & \texttt{foo.mpeg}\\ \hline
    \end{tabular}
    }
  \end{center}
\end{frame}

\begin{frame}[fragile]{Interpreting Special Characters}
  The special characters are
  \begin{center}
    \texttt{\$ * < > \& ? \{ \} [ ]}
  \end{center}
  \begin{itemize}[<+- | alert@+>]
    \item The shell interprets them in a special way uness we escape them (\texttt{\textbackslash\$}), or place
          them in quotes ("\$")
    \item When we first invoke a command, the shell first translates it from a string of characters to a Unix
          command that it understands.
    \item A shell's ability to interpret and expand commands is one of the powers of shell scripting.
    \item These will become your friends, and we'll see them again...
  \end{itemize}
\end{frame}

%%%%%
%%%%%
%%%%%
%%%%%
%%%%% section shell_expansion
%%%%
%%%
%%
%

%
%%
%%%
%%%%
%%%%% section sets_regular_expressions_and_usage
%%%%%
%%%%%
%%%%%
%%%%%
\section{Sets, Regular Expressions, and Usage}
\label{sec:sets_regular_expressions_and_usage}

\begin{frame}[fragile]{\texttt{tr} Revisited}
  \texttt{tr} does not understand regular expressions per se (and really for the task it is designed for they
  don't make sense), but it \colbf{does} understand ranges and \texttt{POSIX} character sets:\wl
  \begin{block}{Useful Sets}
    \begin{itemize}
      \item \texttt{[:alnum:]} - alphanumeric characters
      \item \texttt{[:alpha:]} - alphabetic characters
      \item \texttt{[:digit:]} - digits
      \item \texttt{[:punct:]} - punctuation characters
      \item \texttt{[:lower:]} - lowercase letters
      \item \texttt{[:upper:]} - uppercase letters
      \item \texttt{[:space:]} - whitespace characters
    \end{itemize}
  \end{block}
\end{frame}

\begin{frame}[fragile]{If you Leave this Class with Anything...}
  Quite possibly the two most common things anybody uses in a terminal:

  \begin{itemize}[<+- | alert@+>]
    \item \texttt{find}: searching for files / directories by name or attributes
    \item \texttt{grep}: search contents of files
    \item Used in conjunction with expansions, sets, and regular expressions
  \end{itemize}
\end{frame}

\begin{frame}[fragile]{Finding Yourself}
  \begin{block}{\colbf{find}}
    \texttt{find [where to look] criteria [what to do]}
    \begin{itemize}
      \item Used to locate files or directories
      \item Search any set of directories for files that match a criteria
      \item Search by name, owner, group, type, permissions, last modification date, and \emph{more}
      \item Search is recursive (will search all subdirectories too)
      \begin{itemize}
        \item Sometimes you may need to limit the depth
      \end{itemize}
    \end{itemize}
  \end{block}
\end{frame}

\begin{frame}[fragile]{Some Find Options}
  \begin{itemize}
    \item \texttt{-name}: name of file or directory to look for
    \item \texttt{-maxdepth num}: descend at most num levels of directories while searching
    \item \texttt{-mindepth num}: descend at least num levels of directories while searching
    \item \texttt{-amin n}: file last access was n minutes ago
    \item \texttt{-atime n}: file last access was n days ago
    \item \texttt{-group name}: file belongs to group name
    \item \texttt{-path pattern}: file name matches shell pattern pattern
    \item \texttt{-perm mode}: file permission bits are set to mode
  \end{itemize}

  Of course...a lot more in \colbf{man find}.
\end{frame}

\begin{frame}[fragile]{Some Details}
  \begin{itemize}[<+- | alert@+>]
    \item This command is extremely powerful...but can be a little verbose.  That's normal.
    \item Normally all modifiers for \texttt{find} are evalulated in conjunction (a.k.a AND).  You can condition
          your arguments with an OR by passing the \texttt{-o} flag \emph{for each} modifier you want to be an OR.
    \item You can execute a command on found files / directories by using the \texttt{-exec} modifier, and \texttt{find}
          will execute the command for you.
    \begin{itemize}[<+- | alert@+>]
      \item The variable name is \texttt{\{\}}
      \item You have to end the command with either a
      \begin{itemize}[<+- | alert@+>]
        \item \texttt{;} to execute the command on each individual result
        \item \texttt{+} to execute on all results
        \item \colbf{Note}: You have usually to escape them, e.g. \texttt{\textbackslash ;} and \texttt{\textbackslash +}
      \end{itemize}
    \end{itemize}
  \end{itemize}
\end{frame}

\begin{frame}[fragile]{Some Examples}
  \begin{block}{Find all files accessed at most 10 minutes ago}
    \texttt{find . -amin -10}
  \end{block}
  \begin{block}{Find all files accessed at least 10 minutes ago}
    \texttt{find . -amin +10}
  \end{block}
  \begin{block}{Display all the contents of files accessed in the last 10 minutes}
    \texttt{find . -amin -10 -exec cat {} +}
  \end{block}
  \begin{block}{Accidentally did \texttt{git add} on a Mac and ended up with \texttt{.DS\_Store} Everywhere? }
    \texttt{find . -name .DS\_Store -exec git rm -rf {} \;}
  \end{block}
\end{frame}

\begin{frame}[fragile]{Time for the Magic}
  \begin{block}{\colbf{G}lobally Search a \colbf{R}egular \colbf{E}xpression and \colbf{P}rint}
    \texttt{grep <pattern> [input]}
    \begin{itemize}
      \item Searches \texttt{input} for all lines containing \texttt{pattern}
      \item Can be as easy as specifying a \texttt{string} you need to find in a \texttt{file}
      \item Or it can be much more.
      \item Common: \texttt{<some\_command> | grep <thing you need to find>}
    \end{itemize}
  \end{block}\wl

  Understanding how to use grep is really going to save you a lot of time in the future!
\end{frame}

\begin{frame}[fragile]{Grep Options}
  \begin{itemize}
    \item \texttt{-i}: ignores case
    \item \texttt{-A 20 -B 10}: prints the 10 lines before and 20 lines after each match
    \item \texttt{-v}: inverts the match
    \item \texttt{-o}: shows only the matched substring
    \item \texttt{-n}: displays the line number
    \item \texttt{-H}: print the filename
    \item \texttt{--exclude <glob>}: ignore \texttt{glob} e.g. \texttt{--exclude *.o}
    \item \texttt{-r}: recursive, search subdirectories too.
    \begin{itemize}
      \item \colbf{Note}: you're Unix version may differentiate between \texttt{-r} and \texttt{-R}, check the
            \texttt{man} page.  We'll cover what that means soon.
    \end{itemize}
  \end{itemize}
\end{frame}

\begin{frame}[fragile]{Regular Expressions}
  \begin{itemize}[<+- | alert@+>]
    \item \texttt{grep}, like many programs, takes in a \colbf{regular expression} as its \texttt{input}.
          Pattern matching with regular expressions is more sophisticated than shell expansions, and also uses
          different syntax.
    \item More precisely, a regular expression is a set of strings \-- these strings \colbf{match} the specified
          expression.
    \item When we use regular expressions, it is (usually) best to enclose them in quotes to stop the shell from
          expanding it before passing it to \texttt{grep} / other tools.
  \end{itemize}
\end{frame}

\begin{frame}[fragile]{Regular Expression Notes}
  Some \texttt{regex} patterns perform the same tasks as the wildcards we learned:
  \begin{block}{Single Characters}
    Wild card: \texttt{?} \hspace*{2ex} Regex: \texttt{.}
    \begin{itemize}
      \item Matches any single character.
    \end{itemize}
    Wild card: \texttt{[a-z]} \hspace*{2ex} Regex: \texttt{[a-z]}
    \begin{itemize}
      \item Matches one of the indicated characters
      \item Don't separate multiple characters with commas in the \texttt{regex} form (e.g. \texttt{[a,b,q-v]}
            becomes \texttt{[abq-v]})
    \end{itemize}
  \end{block}
  \begin{block}{A Simple Example}
    \texttt{grep 't.a'} \-- prints lines with things like \texttt{tea}, \texttt{taa}, and \texttt{steap}
  \end{block}
\end{frame}

\begin{frame}[fragile]{A Note on Ranges}
  \begin{itemize}[<+- | alert@+>]
    \item Like shell wildcards, regexs are case-sensitive.  What if you want to match any letter, regardless of case?
    \begin{itemize}[<+- | alert@+>]
      \item If you take a look at the ASCII codes I keep mentioning in \cite{ascii}, you will see that the lower case
            letters come after the upper case letters.
      \item You should be careful about trying to do something like \texttt{[a-Z]}.
      \item Instead, just do \texttt{[a-zA-Z]}.
      \item \colbf{Note}: some programs very well could accept the range \texttt{[a-Z]} correctly.
    \end{itemize}
  \end{itemize}
\end{frame}

\begin{frame}[fragile]{Workarounds}
  \begin{itemize}[<+- | alert@+>]
    \item \texttt{grep} accepts the \texttt{POSIX} sets we learned earlier!
    \begin{itemize}[<+- | alert@+>]
      \item e.g. \texttt{ls | grep [[:digit:]]} gives all files with numbers in the filename
    \end{itemize}
    \item \texttt{*} \-- matches 0 or more occurences of the expression
    \item \texttt{\textbackslash ?} \-- matches 0 or 1 occurences of the expression
    \item \texttt{\textbackslash +} \-- matches 1 or more occurences of the expression
    \item Remember that you can flip the expressions with the not signal: \texttt{\^{}}
    \item The \texttt{\$} can be used to match the end of the line
  \end{itemize}
\end{frame}

\begin{frame}[fragile]{To be continued...}
  There's a lot more going on here.  We'll come back to it soon!
\end{frame}

%%%%%
%%%%%
%%%%%
%%%%%
%%%%% section sets_regular_expressions_and_usage
%%%%
%%%
%%
%

%
%%
%%%
%%%%
%%%%% section more_git
%%%%%
%%%%%
%%%%%
%%%%%
\section{More Git}
\label{sec:more_git}

\begin{frame}[fragile]{Syncing a Fork...}
  ...again!
\end{frame}

%%%%%
%%%%%
%%%%%
%%%%%
%%%%% section more_git
%%%%
%%%
%%
%

\begin{frame}[allowframebreaks]{References}
  \bibliography{lecture05/references}
  \bibliographystyle{abbrv}
\end{frame}

\end{document}
